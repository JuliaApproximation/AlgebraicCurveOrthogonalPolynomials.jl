\documentclass[12pt,a4paper]{article}
\usepackage{amsmath}
\usepackage{amssymb}
\usepackage{graphicx}


\def\addtab#1={#1\;&=}

\def\meeq#1{\def\ccr{\\\addtab}
%\tabskip=\@centering
 \begin{align*}
 \addtab#1
 \end{align*}
  }  
  
  \def\leqaddtab#1\leq{#1\;&\leq}
  \def\mleeq#1{\def\ccr{\\\addtab}
%\tabskip=\@centering
 \begin{align*}
 \leqaddtab#1
 \end{align*}
  }  


\def\vc#1{\mbox{\boldmath$#1$\unboldmath}}

\def\vcsmall#1{\mbox{\boldmath$\scriptstyle #1$\unboldmath}}

\def\vczero{{\mathbf 0}}


%\def\beginlist{\begin{itemize}}
%
%\def\endlist{\end{itemize}}


\def\pr(#1){\left({#1}\right)}
\def\br[#1]{\left[{#1}\right]}
\def\fbr[#1]{\!\left[{#1}\right]}
\def\set#1{\left\{{#1}\right\}}
\def\ip<#1>{\left\langle{#1}\right\rangle}
\def\iip<#1>{\left\langle\!\langle{#1}\right\rangle\!\rangle}

\def\norm#1{\left\| #1 \right\|}

\def\abs#1{\left|{#1}\right|}
\def\fpr(#1){\!\pr({#1})}

\def\Re{{\rm Re}\,}
\def\Im{{\rm Im}\,}

\def\floor#1{\left\lfloor#1\right\rfloor}
\def\ceil#1{\left\lceil#1\right\rceil}


\def\mapengine#1,#2.{\mapfunction{#1}\ifx\void#2\else\mapengine #2.\fi }

\def\map[#1]{\mapengine #1,\void.}

\def\mapenginesep_#1#2,#3.{\mapfunction{#2}\ifx\void#3\else#1\mapengine #3.\fi }

\def\mapsep_#1[#2]{\mapenginesep_{#1}#2,\void.}


\def\vcbr[#1]{\pr(#1)}


\def\bvect[#1,#2]{
{
\def\dots{\cdots}
\def\mapfunction##1{\ | \  ##1}
	\sopmatrix{
		 \,#1\map[#2]\,
	}
}
}

\def\vect[#1]{
{\def\dots{\ldots}
	\vcbr[{#1}]
}}

\def\vectt[#1]{
{\def\dots{\ldots}
	\vect[{#1}]^{\top}
}}

\def\Vectt[#1]{
{
\def\mapfunction##1{##1 \cr} 
\def\dots{\vdots}
	\begin{pmatrix}
		\map[#1]
	\end{pmatrix}
}}



\def\thetaB{\mbox{\boldmath$\theta$}}
\def\zetaB{\mbox{\boldmath$\zeta$}}


\def\newterm#1{{\it #1}\index{#1}}


\def\TT{{\mathbb T}}
\def\C{{\mathbb C}}
\def\R{{\mathbb R}}
\def\II{{\mathbb I}}
\def\F{{\mathcal F}}
\def\E{{\rm e}}
\def\I{{\rm i}}
\def\D{{\rm d}}
\def\dx{\D x}
\def\dy{\D y}
\def\CC{{\cal C}}
\def\DD{{\cal D}}
\def\U{{\mathbb U}}
\def\A{{\cal A}}
\def\K{{\cal K}}
\def\DTU{{\cal D}_{{\rm T} \rightarrow {\rm U}}}
\def\LL{{\cal L}}
\def\B{{\cal B}}
\def\T{{\cal T}}
\def\W{{\cal W}}


\def\tF_#1{{\tt F}_{#1}}
\def\Fm{\tF_m}
\def\Fab{\tF_{\alpha,\beta}}
\def\FC{\T}
\def\FCpmz{\FC^{\pm {\rm z}}}
\def\FCz{\FC^{\rm z}}

\def\tFC_#1{{\tt T}_{#1}}
\def\FCn{\tFC_n}

\def\rmz{{\rm z}}

\def\chapref#1{Chapter~\ref{Chapter:#1}}
\def\secref#1{Section~\ref{Section:#1}}
\def\exref#1{Exercise~\ref{Exercise:#1}}
\def\lmref#1{Lemma~\ref{Lemma:#1}}
\def\propref#1{Proposition~\ref{Proposition:#1}}
\def\warnref#1{Warning~\ref{Warning:#1}}
\def\thref#1{Theorem~\ref{Theorem:#1}}
\def\defref#1{Definition~\ref{Definition:#1}}
\def\probref#1{Problem~\ref{Problem:#1}}
\def\corref#1{Corollary~\ref{Corollary:#1}}
\def\appref#1{Appendix~\ref{Appendix:#1}}

\def\sgn{{\rm sgn}\,}
\def\Ai{{\rm Ai}\,}
\def\Bi{{\rm Bi}\,}
\def\wind{{\rm wind}\,}
\def\erf{{\rm erf}\,}
\def\erfc{{\rm erfc}\,}
\def\qqquad{\qquad\quad}
\def\qqqquad{\qquad\qquad}


\def\spand{\hbox{ and }}
\def\spodd{\hbox{ odd}}
\def\speven{\hbox{ even}}
\def\qand{\quad\hbox{and}\quad}
\def\qqand{\qquad\hbox{and}\qquad}
\def\qfor{\quad\hbox{for}\quad}
\def\qqfor{\qquad\hbox{for}\qquad}
\def\qas{\quad\hbox{as}\quad}
\def\qqas{\qquad\hbox{as}\qquad}
\def\qor{\quad\hbox{or}\quad}
\def\qqor{\qquad\hbox{or}\qquad}
\def\qqwhere{\qquad\hbox{where}\qquad}



%%% Words

\def\naive{na\"\i ve\xspace}
\def\Jmap{Joukowsky map\xspace}
\def\Mobius{M\"obius\xspace}
\def\Holder{H\"older\xspace}
\def\Mathematica{{\sc Mathematica}\xspace}
\def\apriori{apriori\xspace}
\def\WHf{Weiner--Hopf factorization\xspace}
\def\WHfs{Weiner--Hopf factorizations\xspace}

\def\Jup{J_\uparrow^{-1}}
\def\Jdown{J_\downarrow^{-1}}
\def\Jin{J_+^{-1}}
\def\Jout{J_-^{-1}}



\def\bD{\D\!\!\!^-}




\def\questionequals{= \!\!\!\!\!\!{\scriptstyle ? \atop }\,\,\,}

\def\elll#1{\ell^{\lambda,#1}}
\def\elllp{\ell^{\lambda,p}}
\def\elllRp{\ell^{(\lambda,R),p}}


\def\elllRpz_#1{\ell_{#1{\rm z}}^{(\lambda,R),p}}


\def\sopmatrix#1{\begin{pmatrix}#1\end{pmatrix}}

\def\socases#1{\begin{cases} #1 \end{cases}}


\def\Problem#1#2\par{\begin{problem}\label{pb:#1} #2\end{problem}}
\def\Theorem#1#2\par{\begin{theorem}\label{th:#1} #2\end{theorem}}
\def\Conjecture#1#2\par{\begin{conjecture}\label{conj:#1} #2\end{conjecture}}
\def\Proposition#1#2\par{\begin{proposition}\label{prop:#1} #2\end{proposition}}
\def\Definition#1#2\par{\begin{definition}\label{def:#1} #2\end{definition}}
\def\Corollary#1#2\par{\begin{corollary}\label{cr:#1} #2\end{corollary}}
\def\Lemma#1#2\par{\begin{lemma}\label{lm:#1} #2\end{lemma}}
\def\Example#1#2\par{\begin{example}\label{ex:#1} #2\end{example}}
\def\Remark #1\par{\begin{remark*}#1\end{remark*}}


\def\Proof{\begin{proof}}
\def\mqed{\end{proof}}


\def\Figuretwow[#1,#2]#3#4\par{
\begin{figure}[tb]
\begin{center}{
\includegraphics[width=#3]{Figures/#1}\includegraphics[width=#3]{Figures/#2}}
\end{center}
\caption{#4}\label{fig:#1} 
\end{figure}
}

\topmargin  -18mm    
\textheight 254mm   
\textwidth 169mm    
\oddsidemargin -4mm
\begin{document}
\begin{center}
{\Large \textbf{OPs inside cubic curves}}\\
{\footnotesize \today}
\end{center}

\def\C{\tilde C}

Consider for $t > 1$
$$
\Omega = \{ (x,y) : y^2 \leq \underbrace{(1-x^2) (t-x)}_{\rho(x)}, -1 \leq x \leq 1\}
$$
Define
$$
P_{nk}(x,y) := \C_{n-k}^{(k+1)}(x) \rho(x)^{k/2} P_k(y/\rho(x))
$$
where $P_k$ are Legendre polynomials and $\C_k^{(\lambda)}$ are monic semiclassical ultraspherical polynomials, orthogonal w.r.t. $\rho(x)^{\lambda-1/2}$. Denote its norm-squared as
$$
h_k^{(\lambda)} := \int_{-1}^1 \C_k^{(\lambda)}(x)^2 \rho(x)^{\lambda-1/2} \D x
$$
Note $P_{nk}(x,y)$ are orthogonal w.r.t. 
$$
\ip<f,g> := \int_{-1}^1 \int_{-\rho(x)}^{\rho(x)}  f(x,y) g(x,y) \D y \D x
$$
since, with $t = y/\sqrt{\rho(x)}$ we have
$$
\ip<P_{nk},P_{mj}> = \int_{-1}^1 \C_{n-k}^{(k+1)}(x) \C_{m-j}^{(j+1)}(x)  \rho(x)^{(k+j+1)/2} \D x \int_{-1}^1 P_k(t) P_j(t) \D t
$$
Denote it's norm-squared as
$$
h_{nk}^P := \ip<P_{nk},P_{nk}> = h_{n-k}^{(k+1)} h_k^P
$$
where $h_k^P$ are the Legendre norms.


Let's denote the true OPs as $R$, and we know the first 5:
$$
R_{00} := P_{00} | R_{10} := P_{10}, R_{11} := P_{11} | R_{20} := P_{20}, R_{21} := P_{21}
$$
Here we use a total lexicographical ordering of polynomials whenever we talk about "lower order terms". 

\subsection{Quadratic case}

The catch: $P_{nk}$ does not include all polynomials, so for example $R_{22} \neq P_{22}$ since
$$
P_{22}(x,y) = \rho(x) P_2(y/\rho(x)) = {3 \over 2} y^2 - {\rho(x) \over 2} = -{x^3 \over 2} + {\rm l.o.t}
$$
which is cubic in $x$. Idea is to use 
$$
P_{30}(x,y) = \C_3^{(1)}(x) = x^3 + O(x^2)
$$ 
to kill of the cubic term via
$$
R_{22} := P_{22} + {P_{30} \over 2} = {3\over 2} y^2 + {\rm l.o.t.}
$$
Note this is indeed an OP since for lower order polynomials $R_{kj}$ we have
$$
\ip<R_{22},R_{kj}> = \ip<R_{22},P_{kj}> = \ip<P_{22},P_{kj}> + {\ip<P_{30},P_{kj}> \over 2} = 0.
$$
If we denote
$$
h_{nk}^R := \ip<R_{nk},R_{nk}>
$$
we have
$$
h_{22}^R = h_{22}^P + {h_{30}^P \over 2}
$$

\subsection{Cubic case}

The catch, $R_{30} \neq P_{30}$ since 
$$
\ip<P_{30},R_{22}> = {\ip<P_{30}, P_{30}> \over 2} = {h_{30}^P \over 2}  (= {h_3^{(1)} \over 2})
$$
But the solution is simple:
$$
R_{30} := P_{30} - {h_{30}^P \over 2 h_{22}^R} R_{22}
$$
Fortunately the next term is fine
$$
R_{31} := P_{31}
$$
since all l.o.t. $R_{nk}$ are sums of l.o.t. $P_{nk}$. Now note
$$
P_{32}(x,y) = \C_1^{(2)}(x) \pr( {3 \over 2} y^2 - {\rho(x) \over 2} ) = -{x^4 \over 2} + {\rm l.o.t.}
$$
So
$$
R_{32} := P_{32} + {P_{40} \over 2}.
$$
Finally
$$
P_{33}(x,y) = {5 \over 2} y^3 - {3 \over 2} \rho(x) y = - {3 \over 2} x^3 y + {\rm l.o.t.}
$$
Here we use
$$
P_{41}(x,y) = \C_3^{(2)}(x) y = x^3 y + {\rm l.o.t}
$$
so we define
$$
R_{33} := P_{33} + {3 \over 2} P_{41}
$$

\subsection{Quartic case}

$$
R_{40} := P_{40} - {h_{40}^P \over 2 h_{32}^R} R_{32}
$$

$$
R_{41} := P_{41} - {3 h_{41}^P \over 2 h_{33}^R} R_{33}
$$

$$
R_{42} := P_{42} + {P_{50} \over 2}
$$

$$
R_{43} := P_{43} + {3 \over 2} P_{51}
$$

The next term is the hard one as we now have two "bad" terms:
$$
P_{44}(x,y)  = {35 \over 8} y^4 - {30 \over 8} \rho(x) y^2 + {3 \rho(x)^2 \over 8} = {3 \over 8} x^6 - {3 \over 4} t x^5 - {30 \over 8} x^3 y^2 + {\rm l.o.t.}
$$
We can clearly kill it off using $P_{60}$, $P_{50}$, $P_{51}$ and $P_{52}$ though it seems that the construction becomes increasingly difficult for quintic and higher.


\end{document}